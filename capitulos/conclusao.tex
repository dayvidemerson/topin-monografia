\chapter{Conclusão e Trabalhos Futuros}

Este capítulo apresenta os objetivos atingidos com o desenvolvimento do projeto em questão, bem como demonstra futuras melhorias que podem vir a ser desenvolvidas para a continuação deste trabalho.

\section{Conclusão}

No presente trabalho foi demonstrado todo o processo utilizado para o desenvolvimento da tecnologia utilizada para atender os principais problemas dos passageiros de transporte público da cidade de Picos, onde atualmente não existe nenhum meio de consulta a essas informações, não existe aplicativo móvel, site ou rede social com essas informações, o que ocasiona em problemas quanto a necessidade do usuário de se informar quanto ao uso do transporte público.

Com a finalidade de solucionar o problema em questão foi desenvolvido um sistema que com o auxilio das empresas de transporte público permite por meio de um software, nomeado de Topin, que as empresas gerenciem os dados referentes a prestação do serviço, bem como garante aos passageiros o acesso as informações sobre o uso do transporte público de forma rápida e fácil ao utilizar um aplicativo móvel.

O principal problema encontrado foi o método de coleta das coordenadas geográficas que possibilitam a construção do caminho percorrido pelos transportes para a disponibilização do trajeto no aplicativo, que muitas vezes os motoristas cortavam alguns caminhos que eles devem ir por que dependendo do horário eles presumiam que não haviam ninguém, então preferiam cortar caminho, contudo este problema contou com a ajuda da secretaria de transporte e as autoridades da empresa de transporte público, para evitar tais problemas e o aplicativo se mantivesse correto quanto as informações.

De acordo com as funcionalidades que vieram a ser desenvolvidas conclui-se que o sistema atende as necessidades básicas dos passageiros de transporte público, bem como o gerenciamento das informações das empresas de transporte público e facilita a entrega da mudança dessas informações aos passageiros, porém existem algumas melhorias que podem ser visualizadas no tópico a seguir.

\section{Trabalhos Futuros}

Os trabalhos futuros se referem as atividades a serem executadas em um futuro próximo. Destas atividades as que mais se destacam são:

\begin{lista}
\item Comparar o impacto na experiência dos passageiros antes e depois do uso do aplicativo desenvolvido.
\item Desenvolver um módulo que permita o acompanhamento em tempo real dos transportes coletivos;
\end{lista}