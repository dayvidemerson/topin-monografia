\chapter{Introdução}

Neste capítulo serão apresentados conceitos inerentes ao entendimento do trabalho, a motivação, a justificativa, o problema, os objetivos gerais e específicos, bem como a estrutura do trabalho.

\section{Motivação}

Segundo \citeonline{nascimento-lima} o transporte público urbano é um importante mecanismo para o deslocamento de pessoas dentro das cidades permitindo a mobilidade da população entre várias partes da cidade.

A aplicação de tecnologias da informação para solucionar problemas relacionados ao transporte público estão em constante expansão, conhecidas como Sistemas Inteligentes de Transporte Inteligente (em inglês \textit{Intelligent Transportation System} - ITS). Segundo \citeauthor{associacao-nacional} (\citeyear{associacao-nacional}) os ITS consistem na aplicação de um conjunto de tecnologias em constante evolução a problemas comuns do transporte público, como a falta de informação e de planejamento, como também os congestionamentos, as contingências, etc. 

No Brasil, os ITS estão em estágios iniciais se comparados com países desenvolvidos, porém têm mostrado desenvolvimentos importantes nos últimos anos, especialmente com as realizações de eventos como a Copa do Mundo em 2014 e os Jogos Olímpicos no Rio em 2016\cite{alvarado-campos}. Muitas das soluções desenvolvidas para auxiliar os turistas a se locomover na cidade, bem como acompanhar o andamento destes eventos, foram criadas para dispositivos móveis, devido ao alto índice de uso de \textit{smartphones} no mundo.

Segundo dados da 28ª Pesquisa Anual do Uso de Tecnologia da Informação realizada pela Fundação Getúlio Vargas de São Paulo (FGV-SP) o Brasil terá um \textit{smartphone} em uso por habitante até o final de 2017. A pesquisa nacional sobre os hábitos de utilização da \textit{internet} no Brasil, feita pela Fecomércio-RJ e Instituto Ipsos, mostra que o \textit{smartphone} se consolidou como principal meio para acessar a internet no país, utilizado por 69\% dos internautas em 2016 \cite{agencia-brasil}.

\section{Justificativa}

Picos é a terceira maior cidade do estado do Piauí, com sua população estimada pelo Instituto Brasileiro de Geografia e Estatística \cite{ibge} de 73.414 habitantes, sendo que 79\% da população vive na zona urbana e, com base em dados oferecidos por uma empresa que presta serviço de transporte coletivo na cidade de Picos, são vendidas em média 71.854 passagens mensalmente.

Diversos órgãos gestores de transporte público têm aplicado inúmeros esforços para melhorar a qualidade dos serviços
prestados \cite{associacao-nacional}. A modernização da forma que as informações são entregues aos passageiros são
de suma importância para a garantia de uma boa experiência do usuário.

Após uma série de entrevistas e a aplicação de questionários com usuários de ônibus é notável que eles estão
insatisfeitos com o serviço prestado na região de Picos, principalmente pela falta de conhecimento das linhas,
rotas e horários que a empresa oferece e, com isso, se torna necessário o levantamento dos problemas acerca da
insatisfação do passageiro, como também o desenvolvimento de uma solução que reduza significativamente os
problemas encontrados.

\section{Problematização}

Na cidade de Picos existe uma defasagem na entrega de informações para os passageiros, atualmente não existe canal de comunicação para entrega de informações como rotas, linhas e horários do transporte para a cidade, o que leva os passageiros a aprenderem e compartilharem entre si os horários, o que se torna um grande problema ao usuário do transporte público quando precisar utilizar o transporte em um horário que não esteja acostumado e, por este motivo, faz com que ele fique esperando por mais tempo que o necessário na parada, já que não existe se quer uma folha com os horários nas paradas.

Portanto é perceptível a necessidade de um maior controle sobre as informações e principalmente um meio para compartilhar essas informações com a população com o objetivo de reduzir o tempo de espera em paradas, bem como melhorar a experiência que o passageiro tem para com o uso de transporte público na cidade.

\section{Objetivo Geral}

O objetivo geral deste trabalho de conclusão de curso é coletar dados que possibilitem a criação de um sistema de informação aos usuários de ônibus da cidade de Picos, bem como fornecer um meio para que as empresas prestadoras deste serviço gerenciem os dados informados aos passageiros.

A proposta está focada no problema da insatisfação do passageiro em relação ao uso do transporte público, que se deve a dificuldade que os passageiros possuem em acessar as informações referentes ao transporte público. Para isso, propõe-se o desenvolvimento de uma aplicação para dispositivos móveis que possibilite ao seus usuários visualizar informações que o auxiliem na utilização do transporte público, como também uma plataforma \textit{web} que permita as empresas registrarem os dados do serviço.

\section{Objetivos Específicos}

Para a obtenção do objetivo geral, fez-se necessário a sua divisão em algumas etapas distintas, são estas:

\begin{lista}
\item Identificação dos problemas dos usuários de ônibus da cidade de Picos;
\item Elicitação das necessidades dos usuários que possam ser automatizadas por meio de um sistema computacional;
\item Elaborar uma abordagem que possa reduzir a maior quantidade possível dos problemas encontrados, bem como atenda as necessidades elicitadas;
\item Desenvolver um sistema que atenda a abordagem citada acima;
\item Prover uma ferramenta para gerenciar os dados referentes ao serviço de transporte público;
\item Prover uma ferramenta móvel de acesso aos dados referentes ao uso do transporte público;
\end{lista}

\section{Estrutura do Trabalho}

Este trabalho está separado em seis capítulos: Introdução, Fundamentação Teórica, Estado da Arte, Metodologia,
Topin e Considerações Finais, respectivamente, em que o primeiro capítulo trata-se de definir a motivação, a
justificativa, o problema e os objetivos deste trabalho. No segundo capítulo aborda os conceitos necessários
para a compreensão e implementação deste trabalho. No terceiro capítulo são apresentadas algumas soluções
disponíveis para o problema proposto, bem como uma comparação das funcionalidades disponíveis entre as soluções
encontradas e o sistema implementado no presente trabalho. No quarto capítulo é descrito o método de implementação
e pesquisa utilizado para o desenvolvimento do trabalho em questão. No quinto capítulo é realizada uma descrição
das funcionalidades, como se deve o seu funcionamento, além de mostrar algumas das principais tecnologias
utilizadas para o desenvolvimento da solução proposta. O sexto e último capítulo são apresentadas as considerações
finais e trabalhos futuros.