\begin{resumo}
A cidade de Picos é a terceira maior cidade do estado do Piauí e o transporte público da cidade se tornou um dos principais meios de locomoção para estudantes e trabalhadores na cidade. Apesar da utilização constante do transporte coletivo um problema enfrentado diariamente pelos passageiros é a dificuldade de acesso a informações básicas da prestação do serviço como as linhas e horários disponíveis. O que é um grande problema para passageiros que não utilizam o serviço com frequência ou até mesmo para os passageiros habituais quando necessitam utilizar um serviço em um horário que não está habituado.

Este trabalho visa o desenvolvimento de um sistema de informação com o objetivo de atender as necessidades dos passageiros, e para isso, o mesmo foi dividido em duas aplicações: sistema \textit{web} para uso das empresas de transporte público e um aplicativo móvel para os passageiros, o conjunto dessas soluções foi nomeado de Topin. Este \textit{software} que permite às empresas registrarem informações sobre os transportes públicos, como: linhas, horários, trajetos e pontos de referência, bem como possibilita que os passageiros consultem estas informações, além de outras funcionalidades que utilizam essas informações para melhorar a experiência dos passageiros no uso de transporte público na cidade de Picos.

\vspace{\onelineskip}
\noindent
\textbf{Palavras-chaves}: transporte público. passageiro. empresa de transporte público. sistema de informação, aplicativo móvel.
\end{resumo}

\begin{resumo}[Abstract]
\begin{otherlanguage*}{english}
The city of Picos is the third largest city in the state of Piauí and public transportation has become one of the main means of transportation for students and employees in the city. Despite the constant use of public transport, a problem faced by passengers on a daily basis is the difficulty of accessing basic information on the provision of the service, such as available lines and schedules. which is a major problem for passengers who do not use the service frequently or even for regular passengers when they need to use a service at a schedule they are not used to.

This work aims to develop an information system to meet the needs of passengers, and for this, it was divided into two applications: web system for use by public transport companies and a mobile application for passengers, the set of these solutions was named Topin. This software allows companies to record information on public transport, such as lines, timetables, routes and landmarks, as well as allowing passengers to consult this information, as well as other features that use this information to improve the use of public transport in the city of Picos.

\vspace{\onelineskip}
\noindent
\textbf{Key-words}: public transportation. passenger. public transport company. information system, mobile application.
\end{otherlanguage*}
\end{resumo}